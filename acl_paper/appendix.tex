%
% File acl2020.tex
%
%% Based on the style files for ACL 2020, which were
%% Based on the style files for ACL 2018, NAACL 2018/19, which were
%% Based on the style files for ACL-2015, with some improvements
%%  taken from the NAACL-2016 style
%% Based on the style files for ACL-2014, which were, in turn,
%% based on ACL-2013, ACL-2012, ACL-2011, ACL-2010, ACL-IJCNLP-2009,
%% EACL-2009, IJCNLP-2008...
%% Based on the style files for EACL 2006 by 
%%e.agirre@ehu.es or Sergi.Balari@uab.es
%% and that of ACL 08 by Joakim Nivre and Noah Smith

\documentclass[11pt,a4paper]{article}
\usepackage[hyperref]{acl2020}
\usepackage{amsmath}
\usepackage{times}
\usepackage{subfigure}
\usepackage{latexsym}
\usepackage{booktabs}
\usepackage{enumitem}
\renewcommand{\UrlFont}{\ttfamily\small}
% \hypersetup{draft=true}

% This is not strictly necessary, and may be commented out,
% but it will improve the layout of the manuscript,
% and will typically save some space.
\usepackage{microtype}

% TODO(sst): enter aclpaperid here
%\aclfinalcopy % Uncomment this line for the final submission
%\def\aclpaperid{***} %  Enter the acl Paper ID here

%\setlength\titlebox{5cm}
% You can expand the titlebox if you need extra space
% to show all the authors. Please do not make the titlebox
% smaller than 5cm (the original size); we will check this
% in the camera-ready version and ask you to change it back.

\title{On the spontaneous emergence of discrete and compositional signals \\
Appendix}

% TODO(sst): anonymous, so doesn't matter now, but: is this the right author order?
\author{Nur Lan \\
  Computational Linguistics Lab \\
  Tel Aviv University \\
  \texttt{nurlan@mail.tau.ac.il} \\\And
  Emmanuel Chemla \\
  Laboratoire de Sciences Cognitives et Psycholinguistique \\
  Ecole Normale Sup\'erieure \\
  \texttt{chemla@ens.fr} \\\And
  Shane Steinert-Threlkeld \\
  Department of Linguistics \\
  University of Washington \\
  \texttt{shanest@uw.edu}
  }

\date{}

\usepackage{multicol}  % for big figures
\usepackage{amsmath}
\usepackage{graphicx}

\DeclareMathOperator*{\argmax}{arg\,max}
\DeclareMathOperator*{\argmin}{arg\,min}

\newcommand{\changeEC}[2]{{\leavevmode\color{gray}{\scriptsize{#1}}~\color{blue}#2}}
\newcommand{\nbEC}[1]{{\leavevmode\color{blue}{\scriptsize#1}}}
\newcommand{\addEC}[1]{{\leavevmode\color{blue}#1}}

\newcommand{\changeNL}[2]{{\leavevmode\color{gray}{\scriptsize{#1}}~\color{red}#2}}
\newcommand{\nbNL}[1]{{\leavevmode\color{red}{\scriptsize#1}}}
\newcommand{\addNL}[1]{{\leavevmode\color{red}#1}}

\newcommand{\changeSST}[2]{{\leavevmode\color{gray}{\scriptsize{#1}}~\color{violet}#2}}
\newcommand{\nbSST}[1]{{\leavevmode\color{violet}{\scriptsize#1}}}
\newcommand{\addSST}[1]{{\leavevmode\color{violet}#1}}


%% Hide comments:
%\renewcommand{\changeEC}[2]{#2}
%\renewcommand{\nbEC}[1]{}
%\renewcommand{\addEC}[1]{#1}
%
%\renewcommand{\changeNL}[2]{#2}
%\renewcommand{\nbNL}[1]{}
%\renewcommand{\addNL}[1]{#1}
%
%\renewcommand{\changeSST}[2]{#2}
%\renewcommand{\nbSST}[1]{}
%\renewcommand{\addSST}[1]{#1}


\begin{document}

\maketitle


\appendix

\section{Compositionality Test Results}

\label{appendix-comp}

\newcommand{\MESS}[1]{\textsc{m}(#1)}

\begin{table}[hbt!]
\begin{tabular}{lcccc}
\toprule
                       & \multicolumn{2}{c}{\textbf{Compositionality by Addition}} & \multicolumn{2}{c}{\textbf{Composition Network}} \\ \midrule
                       & \textbf{Shared} & \textbf{Non-shared} & \textbf{Shared} & \textbf{Non-shared} \\ \midrule
\textbf{Strict} & \\ 
\emph{10 objects} &  $7.82\%\pm 2.40$  & $11.94\%\pm 2.13$ &  $13.70\%\pm 6.85$  & $10.18\%\pm 6.15$      \\ % \midrule
\textbf{Non-strict} & \\
\emph{5 objects} &
	$16.86\%\pm 3.23$  & $17.14\%\pm 3.54$ &  $15.10\%\pm 2.05$  & $14.35\%\pm 2.74$      \\
%\textbf{Non-strict} & \\
\emph{10 objects} &
	$5.82\%\pm 2.37$  & $6.46\%\pm 1.79$ &  $5.00\%\pm 2.62$  & $5.92\%\pm 2.12$     \\
\emph{15 objects} &
	$3.72\%\pm 1.42$  & $4.00\%\pm 1.54$ &  $1.59\%\pm 1.31$  & $2.48\%\pm 1.05$     \\ \bottomrule
\end{tabular}
\caption*{Communicative success using messages `inferred' by assuming a systemic relation within $\argmin_i$/$\argmax_i$ message pairs. The `compositionality by addition` method assumes that \MESS{$c, \argmax_i$} = \MESS{$c, \argmax_j$} - \MESS{$c, \argmin_j$} + \MESS{$c, \argmin_i$}. The `compositional network' is an MLP trained to predict \MESS{$c, \argmax_i$} from the other three messages. Displayed values are object recovery accuracies averaged for all $i$. }\label{tab:composition_accuracy}
\end{table}


\begin{table}[hbt!]
\begin{tabular}{lcccc}
\toprule
                       & \multicolumn{2}{c}{\textbf{Compositionality by Addition}} & \multicolumn{2}{c}{\textbf{Composition Network}} \\ \midrule
                       & \textbf{Shared} & \textbf{Non-shared} & \textbf{Shared} & \textbf{Non-shared} \\ \midrule
\textbf{Strict} & \\ 
\emph{10 objects} &  $0.23\pm 0.04$  & $0.26\pm 0.04$ &  $0.10\pm 0.01$  & $0.12\pm 0.01$      \\ % \midrule
\textbf{Non-strict} & \\
\emph{5 objects} &
	$6.01\pm 1.82$  & $4.75\pm 1.06$ &  $1.35\pm 0.20$  & $1.74\pm 0.31$      \\
%\textbf{Non-strict} & \\
\emph{10 objects} &
	$3.88\pm 0.91$  & $4.06\pm 0.83$ &  $1.53\pm 0.15$  & $1.76\pm 0.15$     \\
\emph{15 objects} &
	$3.73\pm 0.45$  & $4.68\pm 0.73$ &  $1.87\pm 0.24$  & $1.98\pm 0.23$     \\ \bottomrule
\end{tabular}
\caption{Average MSE loss of predicted objects using messages generated by the two composition methods described above in Table~\ref{tab:composition_accuracy}}\label{tab:composition_mse_losses}
\end{table}


\end{document}
