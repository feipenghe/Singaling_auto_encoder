%
% File acl2020.tex
%
%% Based on the style files for ACL 2020, which were
%% Based on the style files for ACL 2018, NAACL 2018/19, which were
%% Based on the style files for ACL-2015, with some improvements
%%  taken from the NAACL-2016 style
%% Based on the style files for ACL-2014, which were, in turn,
%% based on ACL-2013, ACL-2012, ACL-2011, ACL-2010, ACL-IJCNLP-2009,
%% EACL-2009, IJCNLP-2008...
%% Based on the style files for EACL 2006 by 
%%e.agirre@ehu.es or Sergi.Balari@uab.es
%% and that of ACL 08 by Joakim Nivre and Noah Smith

\documentclass[11pt,a4paper]{article}
\usepackage[hyperref]{acl2020}
\usepackage{times}
\usepackage{latexsym}
\renewcommand{\UrlFont}{\ttfamily\small}

% This is not strictly necessary, and may be commented out,
% but it will improve the layout of the manuscript,
% and will typically save some space.
\usepackage{microtype}

% TODO(sst): enter aclpaperid here
%\aclfinalcopy % Uncomment this line for the final submission
%\def\aclpaperid{***} %  Enter the acl Paper ID here

%\setlength\titlebox{5cm}
% You can expand the titlebox if you need extra space
% to show all the authors. Please do not make the titlebox
% smaller than 5cm (the original size); we will check this
% in the camera-ready version and ask you to change it back.

\newcommand\BibTeX{B\textsc{ib}\TeX}

\title{The spontaneous emergence of discrete and compositional messages}

% TODO(sst): anonymous, so doesn't matter now, but: is this the right author order?
\author{Nur Lan \\
  Computational Linguistics Lab \\
  Tel Aviv University \\
  \texttt{nurlan@mail.tau.ac.il} \\\And
  Shane Steinert-Threlkeld \\
  Department of Linguistics \\
  University of Washington \\
  \texttt{shanest@uw.edu} \\\And
  Emmanuel Chemla \\
  Laboratoire de Sciences Cognitives et Psycholinguistique \\
  Ecole Normale Sup\'erieure \\
  \texttt{chemla@ens.fr}}

\date{}


\newcommand{\changeEC}[2]{{\leavevmode\color{gray}{\scriptsize{#1}}~\color{blue}#2}}
\newcommand{\nbEC}[1]{{\leavevmode\color{blue}{\scriptsize#1}}}
\newcommand{\addEC}[1]{{\leavevmode\color{blue}#1}}

\newcommand{\changeNL}[2]{{\leavevmode\color{gray}{\scriptsize{#1}}~\color{red}#2}}
\newcommand{\nbNL}[1]{{\leavevmode\color{red}{\scriptsize#1}}}
\newcommand{\addNL}[1]{{\leavevmode\color{red}#1}}

\newcommand{\changeSST}[2]{{\leavevmode\color{gray}{\scriptsize{#1}}~\color{violet}#2}}
\newcommand{\nbSST}[1]{{\leavevmode\color{violet}{\scriptsize#1}}}
\newcommand{\addSST}[1]{{\leavevmode\color{violet}#1}}


\begin{document}

\maketitle
% TODO(sst): short [4] or long [8] paper?

% TODO(sst): write abstract
\begin{abstract}
	blah blah blah
\end{abstract}

\section{Introduction}

In a signalling game, artificial agents communicate to achieve a common goal: a sender sees some piece of information and produces a message, this message is then sent to a receiver that must take some action. If the action is appropriate, the whole communication stream, and in particular the choice of the message, is reinforced. For instance, in a referential game, sender and receiver see a set of objects, and the sender must send a message to the receiver, so that the receiver can pick up the right object, as determined in advance for the sender, but unbeknownst to the receiver. 


\section{Function Games}

We here introduce a general communication game setting, which we call Function Games.  Our games contain three basic components: (i) a set of contexts $C$, (i) a set of actions $A$, (ii) a family of functions $F$, from contexts to actions.  One play of a Function Game game runs as follows:
\begin{enumerate}
	\item Nature chooses $f \in F$ and a context $c \in C$.
	\item Sender sees the context $c$ and $f(c)$. \nbSST{I like f(c) here, but f is a bit more appropriate.  What do you all think?}
	\item Sender sends a message $m$ to Receiver.
	\item Receiver sees \emph{a possibly different} context $c'$ and the message $m$ and chooses an action $a'$.
	\item Both are `rewarded' iff $a' = f(c')$.
\end{enumerate}
Two concrete interpretations will be helpful in illustrating the various components.

\noindent \textbf{Generalized referential games.}  A reference game is one in which Sender tries to get Receiver to pick the correct object out of a given set \citep{Skyrms2010, Lazaridou2017, Lazaridou2018, Havrylov2017, Chaabouni2019a}.  Here, contexts are sets of objects (i.e.\ an $m \times n$ matrix, with $m$ objects represented by $n$ features).  Normally (though we will drop this assumption later), $c' = \texttt{shuffled}(c)$: Sender and Receiver see the same objects, but in a different arrangement. Actions are the objects, and the functions $f \in F$ are \emph{choice functions}: $f(c) \in c$ for every context $c$.

\noindent \textbf{Belief update games.}  Contexts can represent possible belief states for the agents.  Letting $A = C$, the functions will then be `belief update' functions, representing e.g.\ how to update an agent's beliefs in the light of learning a new piece of information. \nbSST{What should we cite here? Something from dynamic semantics?}

\section{Experiment}

\subsection{Model}

\subsection{Game Parameters}

\begin{itemize}
	\item strict vs. non-strict context
		\begin{itemize}
			\item num objects for non-strict
		\end{itemize}
	\item equal vs. not equal
	\item object size (num properties)
	\item latent space (msg) dimension [didn't vary this]
\end{itemize}

\section{Results}

\subsection{Communicative success}

% TODO(sst): maybe this isn't a sub-section, just basic results, then subsections for the analyses

\subsection{Discrete signals}

\subsection{Compositionality}

\section{Discussion}

\section{Conclusion}

% TODO(sst): add if accepted
% \section*{Acknowledgments}

% The acknowledgments should go immediately before the references. Do not number the acknowledgments section.
% Do not include this section when submitting your paper for review.

% TODO(sst): cite my forthcoming philosophy of science paper instead? pre-print is only on my website, so URL would be de-anonymizing
\cite{Steinert-Threlkeld2019}

\bibliographystyle{acl_natbib}
\bibliography{anthology,acl2020}


\end{document}
